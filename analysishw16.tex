\documentclass[10pt]{article}
\setlength\headheight{14.5pt}
\title{Homework}
\author{Frederick Robinson}
\date{28 May 2010}
\usepackage{amsfonts}
\usepackage{amsmath}
\usepackage{fancyhdr}
\usepackage{amsthm}
\pagestyle{fancyplain}

\begin{document}

\lhead{Frederick Robinson}
\rhead{Math 321: Analysis}

   \maketitle

\setcounter{tocdepth}{2} 

\section{Problem 3}
\subsection{Question}
For $f \in L^1[a,b]$, define
\[||f||_* = \int_a^b x^2|f(x)| dx.\]
Show that this is a norm on $L^1[a,b]$.
\subsection{Answer}
We must verify the Triangle Inequality, Positive Homogeneity, and Nonnegativity. Let $f, g \in L^1[a,b]$.

{\bf Triangle Inequality}

\begin{proof}
\[||f+g||_* = \int_a^b x^2 |f(x) + g(x)| dx \]
and by properties of the integral (monotonicity, linearity) we have
\begin{eqnarray*} \int_a^b x^2 |f(x) + g(x)| dx  &\leq& \int_a^b x^2 |f(x)| dx + \int_a^b  x^2 | g(x)|  dx\\
&=&||f||_*+||g||_*
\end{eqnarray*}
and so
\[||f+g||_* \leq = ||f||_*+||g||_*\]
as claimed.
\end{proof}

{\bf Positive Homogeneity }

\begin{proof}
\[||\alpha f||_* = \int_a^b x^2 |\alpha f(x)| dx\]
so by linearity of the integral
\[\int_a^b x^2 |\alpha f(x)| dx  =  |\alpha| \int_a^b x^2 | f(x)| dx = |\alpha| ||f||_*\]
and 
\[||\alpha f||_* = \alpha||f||_*\]
as claimed.
\end{proof}

{\bf Nonnegativity }

\begin{proof}
\[||f||_* = \int_a^b x^2 ||f(x) || dx\]
so, since we are integrating a nonnegative function we get
\[ \int_a^b x^2 ||f(x) || dx \geq\]
and 
\[||f||_*  \geq 0\]
as claimed.

Furthermore, if $f =0$ we have
\[||f||_* = \int_a^b  x^2 f(x) dx = \int_a^b 0 =0 .\]
Conversely, given $||f||_*=0$ for some $f$, we have
\[ ||f||_*=\int_a^b x^2 f(x) dx = 0\]
and by a previous exercise (Problem 9 from homework due 7 May) this implies that $x^2 f(x) =0 $ for almost every $x \in [a,b]$. Since $x \neq 0 $ for $x \neq 0$ we must have $f(x)=0$ for almost every $x \in [a,b]$. We may therefore conclude that $||f||_* = 0$ if and only if $f(x)=0$ almost everywhere in $[a,b]$ as desired.
\end{proof}



\section{Problem 5}
\subsection{Question}
Assume $m(E) < \infty$. For $f \in L^\infty (E)$, show that
\[\lim_{p \to \infty} ||f||_p = ||f||_\infty .\]
\subsection{Answer}
We shall assume that the function $f$ is simple. The extension to arbitrary measurable functions $f$ follows directly\footnote{We can fix some sequence of simple functions which converge to $f$. Since each of these have the claimed property then their limit must as well.} from the simple function approximation theorem.
\begin{proof}
If $M$ is the essential upper bound, then $f(x)=M$ on some set of nonzero measure say $A$. Thus, since
\[||f||_p  =  \left[ \int_E |f|^p \right] ^{1/p}  \]
we have
\[    \left[ m(A) M^p \right] ^{1/p}  \leq ||f||_p \leq    \left[ m(E) M^p \right] ^{1/p}  \]
and so
\[   M \left[ m(A)  \right] ^{1/p}  \leq ||f||_p \leq  M  \left[ m(E) \right] ^{1/p}  .\]
Since in the limit $p \to \infty$ $x^{1/p} =1 $ for any $ x  > 0 $ we have
\[\lim_{p \to \infty} ||f||_p = ||f||_\infty \]
as claimed.
\end{proof}

\section{Problem 6}
\subsection{Question}
For $1 \leq p \leq \infty$, if $q$ is the conjugate of $p$, prove that for any $f \in L^p(E)$:
\[||f||_p = \max_{g \in L^q(E), ||g||_q \leq 1 } \int_E fg .\]
\subsection{Answer}
\begin{proof}We have from the text that the conjugate function $g=f^*$ has the desired properties i.e. $||g||_q \leq 1$ and $||f||_p = \int_E f g$. Therefore it remains only to show that there exists no function $g$ such that $||g||_q \leq 1$ and $\int_E fg > ||f||_p$. But this is just a trivial consequence of H\"older's Inequality.\end{proof}

\end{document}
