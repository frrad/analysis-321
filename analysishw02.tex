\documentclass[12pt]{article}
\setlength\headheight{14.5pt}
\title{Homework}
\author{Frederick Robinson}
\date{10 January 2010}
\usepackage{amsfonts}
\usepackage{fancyhdr}
\usepackage{amsthm}
\usepackage{setspace}
\pagestyle{fancyplain}

\begin{document}

\lhead{Frederick Robinson}
\rhead{Math 321: Analysis}

   \maketitle

\setcounter{tocdepth}{2} 


\section{Chapter 6}

\subsection{Problem 7}

\subsubsection{Question}
Suppose $f$ is a real function on $(0,1]$ and $f \in \cal{R}$ on $[c,1]$ for every $c>0$. Define
\[\int_0^1 f(x) dx = \lim_{c\to0}\int_c^1 f(x) dx\]
if this limit exists (and is finite).

\textbf{(a)} If $f\in\cal{R}$ on $[0,1]$, show that this definition fo the integral agrees with the old one.

\textbf{(b)} Construct a funciton $f$ such that the above limit exists, although it fails to exist with $|f|$ in place of $f$.

\subsubsection{Answer}
\textbf{(a)} \begin{proof}By page 121 we know that $f$ must be bounded, say by $M$. We need to show that given $\epsilon >0$ we can find some $c$ such that 
\[\int_c^1 f(x) dx \in B_\epsilon \left( \int_0^1 f(x) dx \right)\]
So, by Theorem 6.12 (c) we have 
\[\int_0^1 f(x) dx = \int_0^c f(x) dx + \int_c^1 f(x) dx\]
and
\[\int_0^c f(x) dx \leq M \cdot c\].

Hence, 
\[\int_0^1 f(x) dx \leq M \cdot c + \int_c^1 f(x) dx\]
but since we can choose any $c>0$ and $M$ is fixed we can choose 
\[c = \frac{\epsilon}{2M}\]
which yields
\[\int_0^1 f(x) dx \leq \frac{\epsilon}{2}+ \int_c^1 f(x) dx\]
So, given $\epsilon$ we can always choose a $c$ such that 
\[\int_c^1 f(x) dx \in B_\epsilon \left( \int_0^1 f(x) dx \right)\]
as desired.\end{proof}

\textbf{(b)} Considered the function which is defined to be $n (-1)^n$ on the last $6/(n^2\pi^2)$ of the interval [0,1] and zero at those $f(x)$ where $x=6/(n^2\pi^2)$. This function is well defined, since we know that $\sum_{n=1}^\infty 6/(n^2\pi^2) = 1 $. 

More specifically the function has value $n(-1)^n$ on the open interval from $(p_n=1-\sum_{m=1}^{n-1} 6/(m^2\pi^2),p_{n+1}=1-\sum_{m=1}^n 6/(m^2\pi^2))$

First we evaluate the integral of the function itself. Consider a partitioning of the interval $[0,1]$ at each $p_n \pm \epsilon$ for some $\epsilon >0$

Then, the lower and upper sums corresponding to the intervals of the partition from $p_n - \epsilon$ to $p_{n+1} + \epsilon$ are the same, since the function is constant valued on these intervals. Moreover, as $\epsilon \to 0$ the value of the upper and lower sums both approach $n (-1)^n(p_{n+1}-p_n)$.

Thus we can express the value of the integral as the sum of the series
\[\sum_{n=1}^\infty \left( \frac{6}{n^2\pi^2}\right)n (-1)^n = \sum_{n=1}^\infty \left( \frac{(-1)^n 6}{n\pi^2}\right)  \]
\[ =\frac{6}{\pi^2} \sum_{n=1}^\infty \left( \frac{(-1)^n }{n}\right)  \]
but we recognize this sum as just a constant multiple of the alternating harmonic series. Hence, the integral converges.

Now we examine the integral of the absolute value of the function. We argue similarly to the above, again partitioning the function at $p_n\pm \epsilon$ as defined above. The difference is that now, as we let $\epsilon \to 0$ the upper and lower sums both go to 
\[\sum_{n=1}^\infty \left( \frac{6}{n^2\pi^2}\right)n  = \sum_{n=1}^\infty \left( \frac{ 6}{n\pi^2}\right)  \]
\[  = \frac{6}{\pi^2} \sum_{n=1}^\infty  \frac{ 1}{n} \]
and so the integral does not exist, as this is the harmonic series, which does not converge.

In the above proof of divergence the important point is that the lower sums diverge. The fact that the upper sums diverge is an immediate consequence of this.

So, we have demonstrated a function whose integral converges, but does not converge absolutely as desired.

\subsection{Problem 8}

\subsubsection{Question}
Suppose $f \in \mathcal{R}$ on $[a,b]$ for every $b>a$ where $a$ is fixed. Define
\[\int_a^\infty f(x) dx = \lim_{b \to \infty}{\int_a^b f(x) dx}\]
if this limit exists (and is finite). In that case, we say that the integral on the left \emph{converges}. If it also converges after $f$ has been n replaced by $|f|$, it is said to converge \emph{absolutely}.

Assume that $f(x) \geq 0$ and htat $f$ decreases monotonically on $[1, \infty)$. Prove that
\[\int_1^\infty f(x) dx\]
converges if and only if 
\[\sum_{n=1}^\infty f(n)\]
converges. (This is the so-called "integral test" for convergence of series.)
\subsubsection{Answer}
\begin{proof}
We begin by showing ($\Rightarrow$) that 
\[\int_1^\infty f(x) dx\]
converges if 
\[\sum_{n=1}^\infty f(n)\]
converges.

So, we assume to start that $\sum_{n=1}^\infty f(n)$ converges. Now consider the partition $P=\{p_n\ |\ p_n=n, n \in \mathbb{N}\}$. Since $f(x)$ decreases monotonically it must be that $inf\{f([p_n,p_{n+1}])\}=f(p_{n+1})$ and similarly that $sup\{f([p_n,p_{n+1}])\}=f(p_{n})$. Thus, the integral which we are trying to evaluate is bounded above by $\sum_{n=1}^\infty f(n)$ and below by $\sum_{n=2}^\infty f(n)$. 

Now we observe that $\int_a^\infty f(x) dx$ may be written as a sum over the domain as 
\[\sum_{n=1}^\infty \left( \int_{p_n}^{p_{n+1}} f(x) dx \right)\]
We know moreover that each of these integrals exist, by Theorem 6.9. Also, since $f(x)$ is always positive each such integral must be positive. Therefore, the integral may be expressed as a sum of a nonnegative series which is bounded above. Hence, by Theorem 3.24 the integral exists.

Now we prove ($\Leftarrow$) that if
\[\int_1^\infty f(x) dx\]
converges then 
\[\sum_{n=1}^\infty f(n)\]
converges.

So assume now that $\int_1^\infty f(x) dx$ converges. Then we can prove that the summation $\sum_{n=1}^\infty f(n)$ satisfies the Cauchy criterion. We established above $\int_k^\infty f(x) dx$ is bounded above by $\sum_{n=k}^\infty f(n)$ and below by $\sum_{n=K+1}^\infty f(n)$. This implies that given a sum $\sum_{n=K+1}^\infty f(n)$ it is bounded above by the integral $\int_k^\infty f(x) dx$. Moreover, since the integral  $\int_k^\infty f(x) dx$ exists and $f$ is nonegative we know that it has the property given $\epsilon>0\  \exists M$ such that $\int_M^\infty f(x) dx< \epsilon$. For otherwise the integral would not exist and instead tend to infinity.

So now we can apply the Cauchy criterion for series. Since an upper bound of the series has the property that given $\epsilon>0\  \exists M$ such that $\sum_M^\infty f(x) < \epsilon$. So must the series itself have this property.

Thus, the sum converges as desired.

\end{proof}
\subsection{Problem 10}

\subsubsection{Question}
Let $p$ and $q$ be positive real numbers such that 
\[\frac{1}{p}+\frac{1}{q}=1\]
\subsubsection{Answer}
\textbf{(a)} We will prove that If  $u\geq 0$ and $v \geq 0$ then 
\[uv \leq \frac{u^p}{p}+\frac{v^q}{q}\]
and that equality holds if and only if $u^p=v^q$
\begin{proof}
We begin by proving the special case of equality

Assume that $u^p=v^q$. 
\[\Leftrightarrow u=v^{q/p}\]
\[\Leftrightarrow vu=v^{q/p+1}\]
\[\Leftrightarrow vu=v^{q/p+1}\]
\[\Leftrightarrow vu =v^{q(1/p+1/q)}\]
\[\Leftrightarrow vu=v^q\]
(Similarly we can show that $vu=u^p \Leftrightarrow u^p=v^q $.)
Thus, $vu=v^q \Leftrightarrow u^p=v^q $ and we see moreover that 
\[uv = \frac{u^p}{p}+\frac{v^q}{q} \Leftarrow vu=v^q\]
since in this case we have 
\[uv = v^q \left( \frac{1}{p}+\frac{1}{q} \right) \checkmark \]
Also, if it is not the case that $vu=v^q$ then it is easy to see that $uv \neq \frac{u^p}{p}+\frac{v^q}{q}$ as for a sum of quotients by $p$ and $q$ to not contain $p$, $q$ we must have the numerators equal.


Now we show that as we vary $u$ we must always have $uv \leq \frac{u^p}{p}+\frac{v^q}{q}$. For, compute the derivative of $uv$ with respect to $u$, and the derivative of $\frac{u^p}{p}+\frac{v^q}{q}$ with respect to $u$. We get $v$ and $u^{p-1}$ respectively. If we have $u^p=v^q$ then these are equal as demonstrated above (we showed that $u v = u^p$ in that case). In the case that $u$ is larger than this value then $u^{p-1}>v$ and in the case that $u$ is less than this value then $u^{p-1}<v$. 

This argument can be repeated in an analogous manner for variations in $v$, and given any $p$ and $q$ we can find values for which $u^p=v^q$.

Thus, we observe that 
\[uv \leq \frac{u^p}{p}+\frac{v^q}{q}\]
as desired\end{proof}


\textbf{(b)} If $f \in \mathcal{R}(\alpha)$, $g \in \mathcal{R}(\alpha)$, $f\geq0$, $g \geq 0$, and
\[\int_a^b f^p d\alpha = 1 = \int_a^b g^q d\alpha, \]
then
\[\int_a^b f g d\alpha \leq 1\]
\begin{proof}

If $0 \leq f \in \mathcal{R} (\alpha) $ and $0 \leq g \in \mathcal{R}(\alpha)$ then $f^p$ and $g^q$ are in $\mathcal{R}(\alpha)$ by Theorem 6.11. Also, we have $f g \in \mathcal{R}(\alpha)$ so we get
\[\int_a^b f g d\alpha \leq \frac{1}{p} \int_a^b f^p d \alpha + \frac{1}{q} \int_a^b g^q d\alpha = 1\]
as desired.\end{proof}


\textbf{(c)} We prove H\"older's inequality
\begin{proof}
If $f$ and $g$ are complex valued then we get 
\[\left|\int_a^b f g d\alpha \right| \leq \int_a^b |f||g| d\alpha.\]

If $\int_a^b|f|^p \neq 0$and $\int_a^b|g|^q \neq 0$ then applying the previous part to the functions $|f|/c$ and $|g|/d$ where $c^p=\int_a^b|g|^q$ and $d^q = \int_a^b|g|^q $ gives what we wanted to show.

\[\left| \int_a^b f g d \alpha  \right| \leq \left( \int_a^b |f|^p d\alpha  \right) ^{1/p}+ \left( \int_a^b |g|^q d\alpha  \right) ^{1/q}\]

However, if one of the above is zero (say without loss of generality 
$\int_a^b|f|^p =0$ then we just have 
\[\int_a^b|f|(c|g|)d\alpha \leq c^q \frac{1}{q}\int_a^b|g|^q d\alpha\]
for $c>0$. Taking the limit $c \to 0 $ we observe that the inequality is still true.

\[\int_a^b |f||g| d\alpha = 0\]

\end{proof}


\end{document}
