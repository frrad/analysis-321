\documentclass[12pt]{article}
\setlength\headheight{14.5pt}
\title{Homework}
\author{Frederick Robinson}
\date{25 January 2010}
\usepackage{amsfonts}
\usepackage{fancyhdr}
\usepackage{amsthm}
\usepackage{amsmath}
\usepackage{setspace}
\newtheorem{theorem}{Theorem}
\newtheorem{lemma}[theorem]{Lemma}
\pagestyle{fancyplain}

\begin{document}

\lhead{Frederick Robinson}
\rhead{Math 321: Analysis}

   \maketitle

\setcounter{tocdepth}{2} 


\section{Chapter 7}
\subsection{Problem 6}

\subsubsection{Question}
Prove that the series 
\[\sum_{n=1}^\infty (-1)^n \frac{x^2+n}{n^2} \]
converges uniformly in every bounded interval, but does not converge absolutely for any value of $x$.
\subsubsection{Answer}
The above series converges uniformly on every bounded interval $[a,b]$

\begin{proof}
\[\sum_{n=1}^\infty (-1)^n \frac{x^2+n}{n^2} = \sum_{n=1}^\infty \left( (-1)^n \frac{x^2}{n^2} +  \frac{(-1)^n }{n} \right)  = \sum_{n=1}^\infty (-1)^n \frac{x^2}{n^2} +\sum_{n=1}^\infty  \frac{(-1)^n }{n} \]
Now we just observe that we know already the alternating harmonic series 
\[\sum_{n=1}^\infty  \frac{(-1)^n }{n} \]
converges. So, taken as a series of functions this converges uniformly. (\emph{Proof:} Weierstrass M-Test with itself.)

It is moreover easy to see that on a bounded interval $[a,b]$ the series 
\[ \left| (-1)^n \frac{x^2}{n^2} \right| \leq  \frac{b^2}{n^2}\]
So this component converges uniformly by the Weierstrass M-Test.

Since each component taken individually converges uniformly it must be that the entire sequence converges uniformly, and we have shown that 
\[\sum_{n=1}^\infty (-1)^n \frac{x^2+n}{n^2} \]
converges uniformly, as desired.
\end{proof}

Now we will show that the above sum does not converge absolutely for any value of $x$.
\begin{proof}

If we fix a value of $x$ the absolute value of the sum becomes 
\[\sum_{n=1}^\infty \left( \frac{x^2}{n^2}+ \frac{1}{n} \right) \]
but of course, for any value of $x$, $n$ we have 
\[ \left( \frac{x^2}{n^2}+ \frac{1}{n} \right) \geq \frac{1}{n}\]
since $x^2$ and $n^2$ are both positive.

Hence, the sum
\[\sum_{n=1}^\infty \left( \frac{x^2}{n^2}+ \frac{1}{n} \right) \]
diverges by comparison with 
\[\sum_{n=1}^\infty \frac{1}{n} \]


\end{proof}

\subsection{Problem 7}

\subsubsection{Question}
For $n= 1,2,3,\dots$, $x$ real, put
\[f_n(x)= \frac{x}{1+n x^2}.\]
Show that $\{f_n\}$ converges uniformly to a function $f$, and that the equation 
\[f'(x)= \lim_{n \to \infty} f'_n(x)\]
is correct if $x \neq 0$, but false if $ x = 0$.
\subsubsection{Answer}
I claim that $\{f_n\}$ converges uniformly to $f(x) = 0$
\begin{proof}
First we establish that $\{f_n\}$ converges pointwise to $0$. Fixing an $x$ we see that
\[ \left| \frac{x}{1+nx^2} \right| \leq \left| \frac{x}{n x^2} \right| \leq \left| \frac{1}{x}\right| \frac{1}{n} \]
and so by comparison test we have pointwise convergence to $0$.

Now, for a fixed $n$ we observe that $\lim_{x \to \infty} f_n(x) = 0$. Since
\[ \left| \frac{x}{1+nx^2}\right| \leq \left|  \frac{x}{nx^2}\right| = \frac{1}{n} \left| \frac{1}{x}  \right| \]
and for fixed $n$ this last value clearly goes to $0$ as $|x|\to \infty$ .

Since this is the case and each $f_n$ is smooth the extrema of some $f_n$ must occur at a point where the derivative is $0$. In particular
\[\frac{d}{dx} \left( \frac{x}{1+ n x^2} \right) = \frac{1-n x^2}{\left(1+n x^2\right)^2}\]
and
\[\frac{1-n x^2}{\left(1+n x^2\right)^2}=0 \Leftrightarrow x = \pm \frac{1}{\sqrt{n}}. \]
Evaluating at these points reveals
\[f_n\left(\pm \frac{1}{\sqrt{n}} \right) = \pm \frac{1}{2\sqrt{n}} .\]
Since these are the extrema of the function we have that 
\[\sup \left| f_n(x) - f(x)\right| = \frac{1}{2\sqrt{n}}.\]
Since $\lim_{n\to \infty} (2 \sqrt{n})^{-1}=0$ we have that $\{f_n\}$ converges uniformly by Theorem 7.9.


\end{proof}


\subsection{Problem 10}

\subsubsection{Question}
Letting $(x)$ denote the fractional part of the real number $x$ (see Exercise 16, Chap. 4, for the definition), consider the function
\[f(x)= \sum_{n=1}^\infty \frac{(n x)}{n^2} \quad (x\mathrm{\ real}).\]
Find all discontinuities of $f$, and show that they form a countable dense set. Show that $f$ is nevertheless Riemann-integrable on every bounded interval.
\subsubsection{Answer}
It will be useful to have a minor extension of Theorem 7.11
\begin{lemma} \label{magic} If $f_n \to f$ uniformly on a set $E \subset X$ and $x$ is a limit point of $E$ such that 
\[\lim_{t \to x^+} f_n(t) = A_n\]
then 
\[ \lim_{t \to x^+} f(t) = \lim_{n\to \infty}A_n  \]
and similarly for left handed limits.
 \end{lemma}

\begin{proof}
Construct a new series of functions say $g_n$ from $f_n$ defined by 
\[
g_n(t) = \left\{
\begin{array}{lr}
A_n& t<x\\
f_n(t)& \mathrm{for\ } t>x
\end{array}
\right.
\]
The right handed limit at $x$ depends only on the function for values greater than $x$, so
\[ \lim_{t \to x^+} f(t) = \lim_{t \to x^+} g(t)  . \]
However, since $f_n$ converges uniformly, and $g_n$ is the same as $f_n$ on $t>x$ it converges uniformly for such values. Moreover it converges uniformly for values $t<x$ since $A_n$ converges by uniform convergence of $f_n$.

Since we set the value of $g_n$ to be $\lim_{t \to x^+} g(t)$ on $t<x$ we have $\lim_{t \to x^+} g(t) = \lim_{t \to x^-} g(t) = \lim_{t \to x} g(t)$. By Theorem 7.11 $\lim_{t \to x} g(t)$ exists and is $\lim_{n \to \infty} A_n$. Hence, so does $\lim_{t \to x^+} f(t)$ and it is exactly $\lim_{n \to \infty}  A_n$ as claimed.



The proof for left handed limits follows similarly.
\end{proof}

For convenience we say
\[f_n(x)=\frac{(nx)}{n^2} \quad \mathrm{and} \quad F_N(x)=\sum_{n=1}^N f_n(x)\]

We observe that $\{F_n\}$ converges uniformly since, 
\[ \left| \frac{(n x)}{n^2} \right| \leq \frac{1}{n^2}.\]
So, as
\[\sum_{n=1}^\infty \frac{1}{n^2}\]
converges, $\{F_n\}$ converges uniformly by Weierstrass M-Test.

I claim that $f$ is discontinuous precisely on $\mathbb{Q} \backslash \{0\}$.
\begin{proof}
We will prove this for $x \in \mathbb{R}^+$ only, however the proof follows almost identically for $x \in \mathbb{R}^- $ and $0$.

For any $m \in \mathbb{N}$ we know that on $[m/n,(m+1)/n)$
\[f_n(x)=\frac{(n x)}{n^2} =  \frac{n x -m}{n^2}.\]
(This follows directly from the definition of the fractional part of a real number.)
Thus, it is clear that either
\[\lim_{t \to x^-}f_n(t) = f_n(x)=\lim_{t \to x^+}f_n(t) \quad \mathrm{or} \quad \lim_{t \to x^-}f_n(t) =\frac{1}{n^2}\neq 0= \lim_{t \to x^+}f_n(t)\]
with the first being the case when $x \neq m / n $ and the second occurring when $x = m/n$.

Since the sum of the limit of two functions is the limit of their sum each $F_n$ must also have property
\[\lim_{t \to x^-}F_n(t) = F_n(x)=\lim_{t \to x^+}F_n(t) \quad \mathrm{for\ }x\cdot m \notin \mathbb{N}\ m \leq n \] 
\begin{equation}\label{ean} \lim_{t \to x^-}F_n(t) - \lim_{t \to x^+}F_n(t) = \sum_{\substack{x\cdot m \in \mathbb{N} \\  m \leq n}} \frac{1}{m^2} \neq 0  \quad \mathrm{for\ }x\cdot m \in \mathbb{N}\ m \leq n \end{equation}
by induction.

So, as we have already proven uniform convergence, Theorem 7.11 implies that for $x \in \mathbb{R} \backslash \mathbb{Q}$
\[\lim_{n\to \infty}F_n(x) = f(x).\]
This however is just the definition of continuity at $x$. Moreover by Lemma \ref{magic} Fact \ref{ean} implies that for all $x \in \mathbb{Q}$ 
\[ \lim_{t \to x^-}f(t) - \lim_{t \to x^+}f(t) \neq 0.\]
Hence, $f$ is discontinuous precisely on $\mathbb{Q}$ as claimed. 
\end{proof}



The function $f$ is Riemann-integrable in every bounded interval since, each $f_n(x)$ has only finitely many discontinuities in each bounded interval and is therefore Riemann-integrable. Each member of the sequence of partial sums $\{F_n\}$ is therefore also Riemann-integrable, being a finite sum of integrable functions. So, since $F_n(x)$ converges uniformly its limit is integrable as well by Theorem 7.16.

\end{document}
