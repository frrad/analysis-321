  \documentclass[12pt]{article}
\setlength\headheight{14.5pt}
\title{Homework}
\author{Frederick Robinson}
\date{14 April 2010}
\usepackage{amsfonts}
\usepackage{fancyhdr}
\usepackage{amsthm}
\pagestyle{fancyplain}

\begin{document}

\lhead{Frederick Robinson}
\rhead{Math 321: Analysis}

   \maketitle

\setcounter{tocdepth}{2} 


\section{Problem 1}
\subsection{Question}
Let $\{E_k\}_{k=1}^\infty$ be a countable disjoint collection of measurable sets. Prove that for any set $A$.
\[m^*\left( A \cap \bigcup_{k=1}^\infty E_k\right) = \sum_{k=1}^\infty m^*(A \cap E_k).\]
\subsection{Answer}
\begin{proof}
We will begin by assuming that both sides of our (alleged) equality are finite.
By countable subadditivity  we have
\[m^*\left( A \cap \bigcup_{k=1}^\infty E_k\right)  \leq \sum_{k=1}^\infty m^*(A \cap E_k).\]
In order to prove the reverse inequality we will demonstrate in particular that given $\epsilon > 0$
\[m^*\left( A \cap \bigcup_{k=1}^\infty E_k\right) + \epsilon \geq \sum_{k=1}^\infty m^*(A \cap E_k).\]

Recall now that Proposition 6 (Page 36) of the book establishes the finite case of this proposition. In particular we have 
\[m^*\left( A \cap \bigcup_{k=1}^n E_k\right) = \sum_{k=1}^n m^*(A \cap E_k).\]
Furthermore, given any $\epsilon>0$ there exists some $n \in \mathbb{N}$ such that 
\[\sum_{k=1}^\infty m^*(A \cap E_k) \leq \sum_{k=1}^n m^*(A \cap E_k) + \epsilon.\]
Taking these together we see that for any $\epsilon > 0$
\begin{eqnarray*}\sum_{k=1}^\infty m^*(A \cap E_k) &\leq& \sum_{k=1}^n m^*(A \cap E_k) + \epsilon \\ &=& m^*\left( A \cap \bigcup_{k=1}^n E_k\right)  + \epsilon \leq m^*\left( A \cap \bigcup_{k=1}^\infty E_k\right)  + \epsilon \end{eqnarray*}
\[  \Rightarrow m^*\left( A \cap \bigcup_{k=1}^\infty E_k\right)  + \epsilon \geq \sum_{k=1}^\infty m^*(A \cap E_k) \]
as desired.

Lastly we must consider the nonfinite cases. If either $m^*\left( A \cap \bigcup_{k=1}^\infty E_k\right)$ or $\sum_{k=1}^\infty m^*(A \cap E_k)$ is nonfinite then so is the other. In particular, 
\[m^*\left( A \cap \bigcup_{k=1}^\infty E_k\right) = \infty \Rightarrow \sum_{k=1}^\infty m^*(A \cap E_k) = \infty\]
by subadditivity and 
\[ \sum_{k=1}^\infty m^*(A \cap E_k) = \infty \Rightarrow m^*\left( A \cap \bigcup_{k=1}^\infty E_k\right) = \infty \]
since
\[ \sum_{k=1}^\infty m^*(A \cap E_k) = \infty\]
implies that given any $N$ there is some corresponding $m$ such that 
\[ \sum_{k=1}^m m^*(A \cap E_k) = N = m^*\left( A \cap \bigcup_{k=1}^m E_k\right) \]
so
\[ m^*\left( A \cap \bigcup_{k=1}^\infty E_k\right) = \infty\]
as desired.
\end{proof}

\section{Problem 3}
\subsection{Question}
For $\epsilon>0$ and $n \in \mathbb{N}$, let $\mathcal{D}_{\epsilon, N}$ be the set of real numbers $x \in [0,1]$ with the following property: for every integer $q \geq N$ and for every integer $0 \leq p \leq q$:
\[\left| x-\frac{p}{q} \right| \geq \frac{1}{q^{2+\epsilon}}.\]
The number $x \in [0,1]$ is called a \emph{Diophantine number} if $x \in \mathcal{D}_{\epsilon, N}$, for some $\epsilon, N$. A number that is not Diophantine is called \emph{Liouville}. Liouville numbers are well-approximable by rational number, and Diophantine numbers are not.

Prove that for every $\epsilon > 0 $, the set
\[\bigcup_{N=1}^\infty \mathcal{D}_{\epsilon, N}\]
has Lebesgue measure $1$ in $[0,1]$ (that is, its complement has measure 0). Conclude that for almost every $x \in [0,1]$ and for every $\epsilon >0$, there exists a $C > 0$ such that 
\[\left| x-\frac{p}{q} \right| \geq \frac{C}{q^{2+\epsilon}},\]
for all $p,q \in \mathbb{N}$. In particular, the set of Diophantine numbers has full measure (though this statement is even stronger than that).
\subsection{Answer}
\begin{proof}
Fixing $\epsilon$ we have that $x \notin \mathcal{K}_\epsilon = [0,1] \cap \bigcup_{N=1}^\infty \mathcal{D}_{\epsilon, N}$ if and only if for each $q,p \in \mathbb{N}$ with $p \leq q$
\[\left| x-\frac{p}{q} \right| < \frac{1}{q^{2+\epsilon}}.\]
Thus we can cover the complement $\mathcal{K}^C_\epsilon \cap [0,1]$ by a finite union of open balls defined as follows for some $n$: 
\[\bigcup_{k=1}^n B_{\frac{1}{n^{2+\epsilon}}} \frac{k}{n}\]
So, in particular $\mathcal{K}^C_\epsilon \cap [0,1]$ has outer measure less than or equal to 
\[n \frac{1}{n^{2+\epsilon}} = \frac{1}{n^{1+\epsilon}}\]
for some choice of $n$. However, since we can choose $n$ to be arbitrarily large, this quantity can be as small as desired, and the outer measure of $\mathcal{K}_\epsilon^C\cap[0,1]=0$ as desired.

Now we may conclude that for almost every $x \in [0,1]$ and for every $\epsilon >0$, there exists a $C > 0$ such that 
\[\left| x-\frac{p}{q} \right| \geq \frac{C}{q^{2+\epsilon}},\]
for all $p,q \in \mathbb{N}$.
\end{proof}


\section{Problem 5}
\subsection{Question}
Let $E$ be a measurable set, and let $N$ be the nonmeasurable set constructed in class. Prove that if $E \subset N$, then $m(E)=0$.
\subsection{Answer}
\begin{proof}
Suppose towards a contradiction that there exists some measurable $E \subset N$ with nonzero measure. Then consider 
\[F=\bigcup_{q \in [0,1] \cap \mathbb{Q}} E +q\]
However, 
\[0<m(E)\leq1 \quad \mathrm{and} \quad  0< m(F) \leq 2\]
by monotonicity of measure, assumption. However this is a contradiction to the countable additivity of measure for measurable sets.
\end{proof}
\end{document}
